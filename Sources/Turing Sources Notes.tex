\documentclass{article}
\usepackage{biblatex}
\addbibresource{../refs.bib}
\usepackage{hyperref}
\usepackage{ulem}

\DeclareLabelname[movie]{
    \field{director}
    \field{producer}
}

\renewcommand{\thesection}{AMT-\Alph{section}}
\renewcommand{\thesubsection}{AMT-\Alph{section}-\arabic{subsection}}
\renewcommand{\thesubsubsection}{\thesubsection.\roman{subsubsection}}

\begin{document}
	\section{Biographical and Personal Documents}
		\subsection{AMT-A-1}
			\subsubsection{Kings Report\cite{amt-a-1-kings}}
				Appears to be obituaries, potentially by Kings College Cambridge, in 1954.
				Notes Turing as not finding interest in Part I of Tripos.
				Notes his interest in formal logic and ordinal logic.
				Very little on his work in WW2.
				\begin{quotation}
					He returned to Kings in 1938. On the Outbreak of war he was invited to join the foreign Office. For his distinguished services during the war he was awarded the OBE. He then turned to the problems involved in the designand use of electronic computing machines.
				\end{quotation}
				Turing's work with the Manchester Computor is noted. 
				Notes his long distance running.
				Notes 1951 election to F.R.S.
				Notes work on theory of organic growth.
				Little said on his death.
				\begin{quotation}
					[H]e was found dead in his house at Wilmslow on Cheshire.
				\end{quotation}

			\subsubsection{Manchester Guardian 1957\cite{amt-a-1-manchester57}}
				Reports verdict on death of Alan Turing.
				Notes that death was ruled as suicide by cyanide poisoning.
				Evidence comes from bubbling contents of a pan and the smell of almonds.
				Notes that Turing's brother John was hmslef alive. 
				Turing's brother seemed to testify that Alan was in good health and had no financial difficulty.

			\subsubsection{Manchester Guardian Obituary (1954/06/10)}
			
			\subsubsection{Manchester Guardian Appreciation by M.H.A.N.}
				1954/06/11.
				MHAN could easily be Maxwell Herman Alexander Newman, who knew Turing.
				Newman did not work with Turing on Enigma but did work at Bletchley Park at the same time.
				
		\subsection{}
		\subsection{}
		Two identical publications, from May and November of 1972, 18 years after Turing's death and 27 after WW2 ended.

		Details Turing's and John von Neumann's relationship prior to WW2.

	\section{Publications, Lectures, and Talks}

	\section{Unpublished Manuscripts and Drafts}

	\section{Correspondence}
		\subsection{Letters to Mother}
			\subsubsection{}
				Unknown date of sending or reception.
				Copy received January 1954, five months before Turing's death.
				If the letter is indeed from that date, Turing would have (likely) been living just south of Manchester, currently a 3 hour train ride from London (likely more back then).
				Could have taken place while Turing was at Princeton, so between 1936 and 1938.
				Could have taken place while Turing was at Bletchley Park during WW2.
				Could have taken place while Turing lived in Hampton, London.

				Notes a "M of S" document. 
				Only lead goes to Memorandum of Sale document for property transactions.
				Could be the subject, might not.

				Notes dislike of American defnition of `classified'/`declassified'.
				I personally agree wth it.
				Turing could read German.
				I recall seeing an early draft of a paper he wrote in French\cite{amt-k-4}, so that adds intrigue.

				Turing later recalls and recommends a shop in London for buying wedding presents.
				He also notes a man named Fred who has a wife who is a landlady.
				She may be his landlady.

				He concludes his letter with a post-script about an individual named Webbs leaving from wherever Turing is.
				A lack of results about any such `Webbs' at Princeton during Turing's time there (besides a few people in the graduating class and Michael Webb who worked at Princeton and was \textit{born} in 1937) discourages my theory of the timing.
				Perhaps this was written while Turing was at Bletchley (no mention of the war, but not unplausible) or when he lived in Hampton (1945-1947).

			\subsubsection{March 15, 1937}
				Letter from Turing to mother from Princeton Graduate College.

				Mentions the recent fellowshp elections.
				Turing was a Jane Eliza Procter Visiting Fellow in his second year at Princeton.
				Includes people he hopes or wonders if will be selected as fellows (Champerowne and Champ, possibly the same person).

				Also notes a play reading society.
				Sounds fun in my personal opinion.

			\subsubsection{March 29, 1937}
				Letter from Turing to mother from Princeton Graduate College.

				Notes that Champernowne (an economist of Champernown constant\footnote{$C_10 = 0.12345678910111213\dots$} fame) was granted the fellowship at King's.

				Relates wrting down some ideas in logc, believed in the document to be preperatory steps for hs PhD Dissertation ``Systems of logic based on ordinals''.

			\subsubsection{Handwritten Letter}
				Cannot read this presently. 
				Come back to it.

		\subsection{Letters to Max Newman}
			All letters are handwrtten (come back to them) except two copies of an unsent letter to Church.
			Turing's handwritng is \textit{very} difficult to read.

			\subsubsection{}

			\subsubsection{}
			
			\subsubsection{}
			
			\subsubsection{}
			
			\subsubsection{}
			
			\subsubsection{Draft Letter to Church}
				Half handwritten, half typed.

			\subsubsection{Other Draft to Church}

		\subsection{Correspondences with I.J. ``Jack'' Good}
			01-03 appear to be the same letter to I.J. Good\footnote{Wikipedia article on I.J. Good: \url{https://en.wikipedia.org/wiki/I._J._Good}}.
			They have the same text, at least.

			All documents are to Good except the last from Good to Turing.

			\subsubsection{September 18, 1948}
				Three years before Turing's work in mathematical biology began.
				Mentions an estimate of a number of neurons, potentially in the human brain?

				Comments on his chess machine he made with one `Champ'.
				Said `Champ' is most likely D.G. Champernowne, who Turing wrote a chess algorithm with begnning in 1948.

				A metaphor or comment Good made about `thnking in analogies' is made.

				The letter is concluded with what appears to be an identity Good thought of, which Turing connects to Poisson's summation formula and contour integration.

			\subsubsection{Same as prior}
			\subsubsection{Same as prior}
			\subsubsection{July 28, 1948}
				Clarifies Turing's estimation about number of neurons to be between $3 \times 10^8$ and $3 \times 10^9$.
				This estimate is based on the weight of a mouse's brain and the mouse brain's neuron count compared to the human brain's (a simple ratio problem).

			\subsubsection{July 25, 1948 (from Good to Turing)}
				Good recalls a lecturer who estimated the human brain to have two million neurons.
				Good thought the estimate to be vastly too low, explaining the origin of the correspondence. 

				Good also asks Turing how `near' he was to getting into the Olympics. 
				The summer olympics in 1948 were held in London, England.
				Good may have been asking how close Turing was to qualifying for the Olympics, or (less likely) how close he was to getting into a stadium to watch the Olympics that summer.
				In the case of the former, Turing was memorably an avid runner.

		\subsection{Correspondence with Robin O. Gandy}
			Gandy was a doctoral student of Turing's\footnote{Wikipedia article on R.O. Gandy: \url{https://en.wikipedia.org/wiki/Robin_Gandy}}.
			Gandy's work was largely about Recursion Theory, which drew from Turing's work.
			
			All documents are dated to 1953-1954, 0-1 years before Turing's death in 1954.

			All documents are handwritten.

		\subsection{Various Letters to Turing}
			\subsubsection{C.G. Darwin; November 11, 1947}
				Letter from physicist Charles Galton Darwin\footnote{Grandson of Charles Darwin. Wikipedia article: \url{https://en.wikipedia.org/wiki/Charles_Galton_Darwin}}, who was director of the National Physics Labratory at the time.
				Darwin mentions reading a recent ``signed and sent out'' paper by Turing.
				He complements the content, but criticizes the quality of the paper.

			\subsubsection{J.Z. Young; January 13, 1951}
				Handwritten letter from John Z. Young, likely the oxford zoologist\footnote{If so, his Wikipedia article: \url{https://en.wikipedia.org/wiki/John_Zachary_Young}}.
				At the time of writing, Turing was turning towards mathematical biology. 
				
	\printbibliography
\end{document}