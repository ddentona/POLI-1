\documentclass{article}
\usepackage{biblatex}
\addbibresource{../refs.bib}

\DeclareLabelname[movie]{
    \field{director}
    \field{producer}
}

\begin{document}
	\section{AMT-A-1}
		\subsection{Kings Report\cite{amt-a-1-kings}}
			Appears to be obituaries, potentially by Kings College Cambridge, in 1954.
			Notes Turing as not finding interest in Part I of Tripos.
			Notes his interest in formal logic and ordinal logic.
			Very little on his work in WW2.
			\begin{quotation}
				He returned to Kings in 1938. On the Outbreak of war he was invited to join the foreign Office. For his distinguished services during the war he was awarded the OBE. He then turned to the problems involved in the designand use of electronic computing machines.
			\end{quotation}
			Turing's work with the Manchester Computor is noted. 
			Notes his long distance running.
			Notes 1951 election to F.R.S.
			Notes work on theory of organic growth.
			Little said on his death.
			\begin{quotation}
				[H]e was found dead in his house at Wilmslow on Cheshire.
			\end{quotation}

		\subsection{Manchester Guardian 1957\cite{amt-a-1-manchester57}}
			Reports verdict on death of Alan Turing.
			Notes that death was ruled as suicide by cyanide poisoning.
			Evidence comes from bubbling contents of a pan and the smell of almonds.
			Notes that Turing's brother John was hmslef alive. 
			Turing's brother seemed to testify that Alan was in good health and had no financial difficulty.

		\subsection{Manchester Guardian Obituary (1954/06/10)}
		
		\subsection{Manchester Guardian Appreciation by M.H.A.N.}
			1954/06/11.
			MHAN could easily be Maxwell Herman Alexander Newman, who knew Turing.
			Newman did not work with Turing on Enigma but did work at Bletchley Park at the same time.
			
	\section{AMT-A-3}
		Note to self: Look at this, specfically the second half of it.
	\printbibliography
\end{document}