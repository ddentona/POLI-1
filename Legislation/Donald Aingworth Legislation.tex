\documentclass{article}
\usepackage{parskip}
\usepackage{enumitem}

\newlist{level}{enumerate}{10}
\setlist[level]{label*=\arabic*., font=\maybeHeadItem}
\newcommand{\headItem}[1]{\global\isHeadtrue\item {\bfseries #1}}
\newcommand{\maybeHeadItem}{%
    \ifisHead\bfseries\global\isHeadfalse\fi%
}
\newif\ifisHead
\pagenumbering{gobble}

\NewDocumentCommand{\subject}{ m }{%
    \begin{center}
        \bfseries%
        POLI 1 Bill 2: #1%
    \end{center}
}%
\NewDocumentCommand{\name}{ m }{{\bfseries #1}}
\NewDocumentCommand{\email}{ m }{ #1 }

\begin{document}
\subject{A Bill for the Expansion of the House of Representatives}
\begin{level}
    \item[Section 1]    This bill may be referred to as ``The Congressional Expansion Act of 2026''.
    \item[Section 2]    The United States House of Representatives will increase in membership from 438 members to 1438 members between 2032 and 2102.
    \item[Section 3]    The rate of increase will vary increasingly by year.
    \begin{level}
        \item[3.1]  In 2032, the House of Representatives will increase in membership from 438 members to 538 members.
        \item[3.2]  In 2042, the House of Representatives will increase in membership from 538 members to 638 members.
        \item[3.3]  In 2052, the House of Representatives will increase in membership from 638 members to 838 members.
        \item[3.4]  In subsequent years ending in the number 2 and while there are no more than 1437 members of the House of Representatives, the House of Representatives will increase in membership by 200 members.
    \end{level}
    \item[Section 4]    The new seats resultant from this bill will be apportioned as part of the apportionment of seats of the House of Representatives between states subsequent each census.
    \item[Section 5]    This legislation will go into effect on January 1 of 2026.
\end{level}
\begin{center}
    \textit{Presented to the Student Congress by Donald Aingworth (IV) of Wyoklahoma.}
\end{center}

\end{document}