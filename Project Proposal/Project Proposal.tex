\documentclass{article}

\NewDocumentCommand{\subject}{ m }{%
    \begin{center}
        \bfseries%
        #1%
    \end{center}
}%
\NewDocumentCommand{\name}{ m }{{\bfseries #1}}
\NewDocumentCommand{\email}{ m }{ #1 }

\begin{document}
\subject{Doanld Aingworth's Project Proposal for Independent Study for Fall 2025}
\small

This independent study is to provide a graded History paper for submission with an application to Princeton University.
The paper in question should be one to two pages long and should be on a topic that can be simply logically linked to Princeton University but not about Princeton itself.
Furthermore, it would be ideal if said paper would be relatable to my own application to the school.
The three best topic ideas are the Presidential Election of 1912, Kurt Gödel, and Alan Turing's work in early computer science.

\section*{Deliverables}
Only one deliverable is necessary for this independent study: a graded one-to-two-page paper or essay on a subject, ideally among the below listed topics.
This paper should be a history paper, of parameters dictated by the graded paper submission for an application to Princeton University.


\section*{Topic ideas}

\subsection*{1. The United States Presidential Election of 1912}
The 1912 election was among the most chaotic presidential elections. 
On the eve of the first world war, when European tensions were continuously growing, the United States had its presidential election.
Incumbent William Howard Taft was challenged by former president Theodore Roosevelt, eventually splitting the Republican Party in two for the election.
Meanwhile, perennial Democratic candidate William Jennings Bryan stepped back to allow for the rise of New Jersey Governor Woodrow Wilson.
Circumstances of the election brought Wilson's election to the presidency, one of only two Democrats elected to the White House between the Civil War and the Great Depression.

Woodrow Wilson attended college at Princeton University and served as president of the college before becoming Governor of New Jersey.


\subsection*{2. The Life and Work of Mathmatician Kurt Gödel}
Kurt Gödel was one of the most influential mathmaticians of all time. 
Born in Austria ahead of the first world war, Gödel is known as among the most significant logicians in history. 
He is well-known for first his completeness theorem about the provamility of statements in different models.
Later and more famously, he became known for his incompleteness theorems, which established that there will always exist specific statements that are unprovable as either true or false given a specific set of axioms.
When immigrating the United States in the 1940s to escape Nazi Germany's invasion of Austria, he famously discovered an inconsistency in the US Constitution that would have allowed the united States to become a dictatorship, which is still theorized about and became known as ``Gödel's Loophole''. 

After moving to the US in 1939, Gödel accepted a position at the Institute for Advanced Study at Princeton, which he was a full professor at from 1953 until he became professor emeritus in 1976. 
The Institute for Advanced Study at Princeton has no formal relationship with Princeton University, although the two are located in the same city.


\subsection*{3. The Life and Research of Alan Turing}
Alan Turing is generally known as the father of theoretical computer science. 
Born in England in 1912, Turing showed significant ability in mathematics and related fields from a young age. 
In 1936, he reformulated findings by Kurt Gödel on limits and computation to create simple hypothetical devices called ``Turing Machines'', which were proven to be able to perform any conceivable mathematical calculation representable as an algorithm. 
These Turing Machines are the foundations of modern computers.
This in turn was done in parallel to the research of Alonzo Church in Lambda Calculus, who would later be his doctoral advisor when getting his PhD.
During the Second World War, Turing worked for the British Government to devise a machine to crack the German Enigma cipher machine. 
Such events are the subject of \textit{The Imitation Game (2014)}.
After the second world war, Turing did further work on early computers both for the English Government and the University of Manchester, including creating an algorithm capable of playing a game of chess.
He also did some work with mathematical biology, specifically in morphogenesis.
Turing died young, found dead in June 1954 at age 41, confirmed to have died of cyanide poisoning as an act of suicide. 
This came after a 1952 conviction for a homosexual relationship, at the time a criminal act in the United Kingdom, followed by undergoing mandatory chemical castration.

Turing received his PhD from Princeton University under Alonzo Church.
Church also received his Bachelor's degree and PhD from Princeton and would also be a topic that meets the requirements listed in the objective of this independent study project. 

\subsection*{Personal Opinions}
All of the above work as topics I am personally interested in and would like to research.
The Election of 1912 has the most direct connection to Princeton University and likely has the most history-based sources about it, but it is the least associated with my areas of study. 
Kurt Gödel has easily the least direct connection to Princeton University and most information will likely be mathematical rather than historical, although it does connect with my areas of study.
Alan Turing connects very well with my intended areas of study, as well as a much stronger connection to Princeton University than Gödel.
Like Gödel, a paper on Turing would have many fewer pure history sources than other subjects, which could be rectified by a transformation into a paper about the early history of theoretical computer science. 

\begin{center}
    \textit{Donald Aingworth; daingwor@mail.ccsf.edu}
\end{center}
\end{document}