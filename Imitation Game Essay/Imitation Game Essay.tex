\documentclass[11pt]{extarticle}
\usepackage[citestyle=numeric,datamodel=../citationfields]{biblatex}
\addbibresource{../refs.bib}
\usepackage{../citationformats}
\usepackage[a4paper, margin=1in]{geometry}
\usepackage[hidelinks]{hyperref}
\usepackage{bookmark}
\usepackage{setspace}
\usepackage{titlesec}
\usepackage{ulem}

% \DeclareLabelname[letter]{
% 	\field{author}
% 	\field{addressee}
% 	\field{date}
% 	\field{location}
% 	\field{title}
% 	\field{note}
% }
% \DeclareLabelname[movie]{
%     \field{director}
%     \field{producer}
% }

\renewcommand{\thesection}{AMT-\Alph{section}}
\renewcommand{\thesubsection}{AMT-\Alph{section}-\arabic{subsection}}
\renewcommand{\thesubsubsection}{\thesubsection.\roman{subsubsection}}

\setstretch{1.05}

\begin{document}
    {\noindent
    John Doe\\
    Professor Smith\\
    HIST 21: History of Computers\\
    % Donald Aingworth\\
    % Professor Ryer\\
    % HIST 46: Independent Study\\
    \today}

    \begin{center}
        [TITLE]
    \end{center}
    \noindent
    According to a 2024 survey by the Digital Education Council, 86\% of students claim to use Artificial Intelligence in their studies\cite{de-studentsUseAI}, raising the question of how successful this AI is relative to its earliest benchmarks.
    An early test theorized for Artificial Intelligence came from the ``Father of Theoretical Computer Science'' Alan M. Turing in his 1950 paper \textit{Computing Machinery and Intelligence}\cite{amt-b-9}.
    For this test, Turing predicted a machine able to engage in conversation with people.
    \uline{Despite AI's early focus on symbolic logic and later on game playing, recent developments of large language models and conversational AI have brought renewed historical relevance to Turing's philosophical test called the Imitation Game.}

    In order to understand the impact of Turing's philosophical theories, familiarity with the Imitation Game is necessary. 
    Turing's Imitation Game consists of a machine capable of replicating human conversation attempting to convince a person it is more likely to be human than a real person is.
    Conversation about a mechanical mind prior to this was largely philosophical about whether an artificial brain is possible and the existence of a mind outside the brain\cite{rd-meditations}.
    Turing effectively dismissed these questions by claiming that a more appropriate question is whether the machine is able to convince others that it is human.
    
    Despite Turing's emphasis in his paper on testing conversational distinguishability of Artificial Intelligence from humans, early AI focused on symbolic logic and proving theorems. 
    Turing's early work led to the development of symbolic logic machines to an extent.
    Early and modern computers are largely based on Turing Machines, originally outlined in the 1937 paper \textit{On computable numbers, with an application to the Entscheidungsproblem}\cite{amt-b-12}.
    Allen Newell and Herbert Simon's 1956 report \textit{The logic theory machine--A complex information processing system}\cite{diesel-InformationTheory} and the associated program Logic Theorist was an early example of AI that operated on symbolic logic.
    Newell and Simon described the goal of Logic Theorist to be ``a program for constructing chains of theorems, not at random but in response to cues that make discovery of cues possible within a reasonable computing time''(\citeauthor{diesel-InformationTheory}, p.5).
    These innovations were revolutionary, but they strayed from Turing's vision of conversational AI.

    The game-playing machines also played a prominent role in the development of AI decades before the advancements required for conversational AI Turing initially described.
    An example of this is found in chess.
    Chess programs have existed since Turing and Champerowne created the Turochamp algorithm in 1950, which was able to play a game of chess\cite{amt-enigma}.
    Over the next half century, chess programs advanced until the IBM supercomputer Deep Blue won a six-game chess match against world chess champion Garry Kasparov in 1996\cite{bw-deepblue}.
    These had similar concepts to the Turing Machine, in that the programs sought to play the same games that humans did.
    However, they diverged from Turing when the computers' goal was to move beyond the skill capacity of humans.

    Conversational AI also has a long history, despite a slower start.
    An early conversational AI was ELIZA, developed between 1964 and 1967 by Computer Scientist Joseph Weizenbaum.
    ELIZA used pattern matching to simulate a Rogerian Psychoanalyst.

    % In \textit{Computing Machinery and Intelligence}, Alan Turing predicted a dialogue machine which would be able to mimic a human mind.
    % This prediction would be far ahead of its time, as AI/ML initially manifested as symbolic logic machines and eventually game-playing machines decades before Large Language Models achieved prominence.
    % Despite this, Turing developed strong philosophical theories related to whether an artificial mind is possible. 
    % These laid out a target that language models would align themselves towards.
    % This target took the form of the dialogue machine able to convince a human that it is more likely to be a human than another person through a process called ``The Imitation Game''.
    % His prediction of training artificial intelligence described reinforcement learning, which remains a major paradigm of machine learning.
    % This prediction was contemporary to the advent of the artificial neuron and neural network, a major foundation of AI/ML.

    % Arguably the most famous idea in \textit{Computing Machinery and Intelligence} is what Turing calls ``The Imitation Game''. 
    % The Imitation Game is a 
    % Turing uses the Imitation Game as a destination for Artificial Intelligence.

    \printbibliography
\end{document}